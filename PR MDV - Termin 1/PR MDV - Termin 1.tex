\newcommand{\institut}{Institut f\"ur Energie und  Automatisiertungstechnik}
\newcommand{\fachgebiet}{Elektronische Mess- und Diagnosetechnik}
\newcommand{\veranstaltung}{Praktikum Messdatenverarbeitung}
\newcommand{\pdfautor}{\"Ozg\"u Dogan (326 048), Timo Lausen (325 411), Boris Henckell (325 779)}
\newcommand{\autor}{\"Ozg\"u Dogan (326 048)\\ Timo Lausen (325 411)\\ Boris Henckell (325 779)}
\newcommand{\pdftitle}{Praktikum Messdatenverarbeitung  Termin 1}
\newcommand{\prototitle}{Praktikum Messdatenverarbeitung \\ Termin 1}
\newcommand{\aufgabe}{}

\newcommand{\gruppe}{Gruppe: G1 Fr 08-10}
\newcommand{\betreuer}{Betreuer: J\"urgen Funk}

\input{../../packages/tu_header_8}
%---------------------------------------------------------------------
%---------------------------------------------------------------------
%---------------------------------------------------------------------

\section{Vorbereitungsaufgaben}
\begin{quote}
	\subsection{Quellcode}
    \begin{quote}
\begin{lstlisting}
// Vorbereitungsaufgabe Termin 2
// Özgü¸ Dogan (326048)
// Timo Lausen (325411)
// Boris Henckell (325779)


#include <avr/io.h>
#include <avr/interrupt.h>
#include <inttypes.h>

ISR(ADC_vect)
{
uint16_t ADUWERT = ADC;
PORTC &= ~(1<<PC5);     //anschalten der Roten LED
if (ADUWERT<=340){
    PORTC |= (1<<PC1);  //Orangene LED ausgeschaltet
    PORTC |= (1<<PC4);  //Grüne LED ausgeschaltet
    }
else if (ADUWERT<=682){
    PORTC |= (1<<PC1);  //Orangene LED ausgeschaltet
    PORTC &= ~(1<<PC4); //Grüne LED anschalten
    }
else {
    PORTC &= ~(1<<PC1); //Orangene LED anschalten
    PORTC &= ~(1<<PC4); //Grüne LED anschalten
    }

}

int
main (void)
{

    // ADU anschalten

    ADCSRA |= (1<<ADEN);   // shiften 1 in ADCSRA um ADEN, damit ADU angeht

    // ADCH und ADCL Register (je 8bit)
    // 10bit rechts auslesen - ganz normal
    // links adjusten
    // ADMUX |= (1<<ADLAR);   // shiften 1 in ADMUX um ADLAR

    ADMUX  |= (1<<REFS1) | (1<<REFS0);      //Voltage reference 2.56V (s. Datenblatt S.289, Tabelle 26-3)
    ADCSRA |= (1<<ADPS0) | (1<<ADPS2);      //F_adu = (F_clk)/32 (s. Datenblatt S.293, Tabelle 26-5)

    //  Als Eingang soll der Kanal ADC0 im Single-Ended-Modus genutzt werden.
    //  Daf¸r wird im ADCSRB Register MUX 4:0 zu null gesetzt. Der 5. Bit (MUX 5) ist nicht g¸ltig f¸r ATmega1281/2561.
    //  (s. Datenblatt S.290, 26.8.2)

    //  F¸r den free running modus m¸ssen die ADTS2,1,0 Bits im ADCSRB Register null sein
    //  ADCSRB &= ~(1<<ADTS2) & ~(1<<ADTS1) & ~(1<<ADTS0);         (s. Datenblatt S.296, Tabelle 26-6)

    sei();                                  //aktiviert Interrupts allgemein (‰hnlich wie ein Notschalter)
    
    //alternativ Interrupts global aktivieren: //(s. Datenblatt S.14, 7.4.1)
    //SREG |= (1<<I);                           
                                            
    



    ADCSRA |= (1<<ADIE);                    //aktiviert das Interrupt im ADC (s. Datenblatt S.292, 26.8.3)

    while ( 1 ) {

    }
    return 0 ;
}



\end{lstlisting}         
    \end{quote}

    \subsection{Abtastrate des ADU}
    \begin{quote}
        Der ADU arbeitet bei dieser Einstellung mit einer Abtastrate von $\frac{f_{clk}}{32}$. Es lassen sich auch
        Abtastraten von $\frac{f_{clk}}{2}, \frac{f_{clk}}{4}, \frac{f_{clk}}{8}, \frac{f_{clk}}{16},
        \frac{f_{clk}}{64}$ und $\frac{f_{clk}}{128}$ einstellen.\\
        Für eine Umsetzung benötigt der ADU-jedoch $13$ solcher Takte und daher ist die effective Abtstrate $f_{sample}
        = \frac{f_{clk}}{13 \cdot   32}$.\\
        Im Datenblatt ist angegeben, dass ein ADU, der eine kleinere Auflösung
        als 10 Bit benötigt ( in unserem Fall arbeiten wir mit 8 Bit) mit einer
        Clock arbeiten, der eine Frequenz zwischen $50$ und $200 kHz$ hat. Wenn
        wir von $200 kHz$ ausgehen würde die eingestellte Abtastrate mit
        $\frac{200 kHz}{13 \cdot 32} \approx 480 Hz$ betragen.\\
        
        Die Clock-Frequenz kann auch bis zu $1 MHz$ hochgehen, falls eine höhere
        Abtastrate erforderlich ist.
        
        
    \end{quote}

\end{quote}

%--------------------------------------------------------------------
%--------------------------------------------------------------------

\section{Versuch}
\begin{quote}
	
\end{quote}

%--------------------------------------------------------------------
%--------------------------------------------------------------------

\section{Ergebnisse}
\begin{quote}
	
\end{quote}

%--------------------------------------------------------------------
%--------------------------------------------------------------------



\end{document}
